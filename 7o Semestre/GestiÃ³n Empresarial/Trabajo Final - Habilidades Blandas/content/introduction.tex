%INTRODUCCIÓN
%Descripción: En este documento se muestra la introducción y apartado teórico del documento en cuestión.

La época actual, caracterizada por el rápido intercambio de información y difusión de tecnología, el cambio acelerado y la globalización, ha configurado nuevos mercados laborales en los que se pone de manifiesto una escasez de perfiles competitivos \cite{innova:2016,vart:2000}. 
\vspace{5mm}

\noindent Un estudio desarrollado por Manpower sobre la escasez de talento humano en el mercado, afirma que una de las principales dificultades para encontrar candidatos es la falta de habilidades de empleabilidad o \textquote{\textit{soft skills}}: Uno de cada cinco directivos (19\%) afirma que los candidatos carecen de las competencias de empleabilidad (las llamadas habilidades sociales o \textit{soft skills}) requeridas. Los directivos han identificado una serie de déficits en materia de habilidades sociales, entre los que se incluyen el entusiasmo / motivación (5\%), las habilidades de trato interpersonal (4\%), la profesionalidad, es decir, el cuidado por el aspecto, la puntualidad, etc. (4\%), y la
flexibilidad y adaptabilidad (4\%) \cite{mpst:2015}.
\vspace{5mm}

\noindent Mónica Flores, directora general del ManpowerGroup LATAM, indica que el 39\% de las empresas de la región tuvieron dificultad para cubrir vacantes en el período de 2013, por lo que señalan que existe una escasez de talentos en el mercado. La autora, menciona que la brecha entre la oferta y demanda de talentos se debe a muchos factores, entre los que destaca el hecho de que los programas educativos presentan poca articulación con las estrategias empresariales: no se prepara a los jóvenes en las habilidades y las competencias necesarias para un trabajo formal, el aprendizaje suele ser de carácter memorístico y centrado en el maestro. Mientras que el 72\% de unidades educativas señala que sus recién graduados están listos para trabajar, el 42\% y el 45\% de graduados y empleadores señalan que no lo están respectivamente \cite{flores:2014}.
\vspace{5mm}

\noindent El presente artículo pretende identificar los retos que se le presentan al Ingeniero en Sistemas Computacionales en la globalización, pero para ello primero debemos tener claro a qué profesional nos estamos refiriendo, éste es una persona que podrá desempeñarse en equipos multidisciplinarios e interdisciplinarios en los ámbitos del desarrollo de software y hardware, sustentando su actuación profesional en valores éticos y de responsabilidad social, con un enfoque de liderazgo y sostenibilidad en los sectores público y privado \cite{escom:perfil}. 
\vspace{5mm}

\noindent Ahora que tenemos claro quién será el actor, debemos entender el concepto de la globalización donde tendrá que enfrentar estos retos, esto se refiere a un proceso donde ciertas actividades tienen la capacidad de funcionar como unidad en tiempo real a escala planetaria, esto se debe a que se ha creado una red de flujos en la que confluyen las funciones y unidades de todos los ámbitos de la actividad humana, contribuyendo más las dominantes \cite{pnud:1999}. Esto resulta en una competencia donde los entes no participan aisladamente sino que lo hacen en un entorno productivo e institucional que expande sus fronteras acorde a la nueva división internacional creada por este efecto \cite{vban:2000}.
\vspace{5mm}

\noindent Una vez identificado al actor y a su escenario, para analizar su interacción y los retos que se le presentan, se utilizará como estrategia las habilidades blandas de las que puede llegar a disponer el actor si las desarrolla correctamente y cómo puede superar, o cuando menos le facilitará, enfrentarse a dichos retos. Las denominadas habilidades blandas son aquellos atributos o características de una persona que le permiten interactuar con otras de manera efectiva, son el resultado de una combinación de habilidades sociales y por ello tienen relación con lo que se conoce como inteligencia emocional \cite{jml:2019}.
