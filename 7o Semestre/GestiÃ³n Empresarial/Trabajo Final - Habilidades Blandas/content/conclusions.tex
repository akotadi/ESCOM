%CONCLUSIONES
%Descripción: En este documento el alumno aporta una breve reseña a modo de cierre del tema.

Nuestra profesión enfrenta muchos retos derivados de una enseñanza que ha acarreado un ideal donde las personas sólo van a tener que tratar con cosas, dícese de las computadoras, los circuitos, las máquinas, etc. Sin embargo, el mundo ha cambiado y es difícil seguirle el ritmo, no es sólo que debamos mantenernos actualizados en nuestros conocimientos debido a que nuestra área así lo exige, lo que ayer servía probablemente mañana ya esté obsoleto porque esto avanza a pasos agigantados y cada vez es más complicado porque los planes de estudio no se pueden actualizar tan rápido como la industria lo quisiera. 
\vspace{5mm}

\noindent Aunado a esto, ahora se integra otro requisito que son las habilidades blandas que llegan a cambiar todo el paradigma, ahora un ingeniero debe poder comunicarse con cualquier persona en la empresa, desde una persona de intendencia hasta los directivos para explicarles funcionamientos de dispositivos, procedimiento o ideas simplemente. Pero no se queda ahí, gracias a la globalización ya no es sólo explicarlo en nuestro sitio de trabajo, posiblemente debamos hacerlo también con otras filiales que pueden encontrarse en otro punto de la ciudad, en otro estado o incluso en otro país. 
\vspace{5mm}

\noindent Es evidente que la lenta burocracia en nuestro país, así como una ideología casi conservadora que está presente en muchas personas de aquí, no permite que la educación tradicional se adapte a la velocidad suficiente a todos estos cambios requeridos; es por ello que el ingeniero ahora debe destacar en un campo en particular, que es la educación autodidacta, que le permita desarrollar tanto sus habilidades duras como sus habilidades blandas, además de buscar cambiar la ideología que se tiene en este campo de trabajo y dar una mayor importancia a estas últimas habilidades porque el trabajo que antaño se veía como el primordial, digamos: escribir código, diseñar software, armar dispositivos, entre otros; cada vez más se ve relegado a un plano secundario debido a que las máquinas, y en particular la inteligencia artificial, llegan a abarcar esas tareas un tanto triviales y en nosotros queda liderar los equipos que llevan todo este desarrollo.
\vspace{5mm}

\noindent Esto implica que nosotros estamos a cargo del avance de esas tecnologías, y no suena coherente que le enseñemos a una máquina a pensar, a hablar y a sentir cuando nosotros somos visto como máquinas y no hemos desarrollado esas capacidades adecuadamente. Pero la enseñanza de las llamadas \textquote{\textit{soft skills}} facilitarán esos retos a los que nos enfrentamos, desde comunicarnos hasta poder trabajar con equipos multiculturales alrededor del mundo. Porque cada vez se requiere que más y más personas se unan a estas áreas y trabajen en conjunto para lograr avances que hasta este momento no han dejado de impresionarnos porque, desde la humilde opinión del autor, el límite de estos logros está únicamente en nuestra imaginación.