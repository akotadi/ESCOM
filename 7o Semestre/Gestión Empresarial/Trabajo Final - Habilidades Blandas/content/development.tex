%DESARROLLO
%Descripción: En este documento se expande el tema resaltando la situaciones que enfrentará al desempeñar su actividad profesional utilizando como estrategia las habilidades blandas.

\section{Globalización}

En los últimos años el mundo ha cambiado de manera drástica debido a muchos factores, principalmente a los avances científicos y tecnológicos que han permitido que las distancias se vean aminoradas, repercutiendo en cómo percibimos el mundo y sus aparentes fronteras. Las diferencias geográficas, sociales, históricas, culturales se han aminorado dando lugar a la globalización que ha integrado a todo el mundo siendo difícil marginarse ante este proceso.
\vspace{5mm}

\noindent Mientras algunos países han optado por intentar mantenerse alejados de este proceso, o cuando menos aminorar su impacto, América Latina busca integrarse para desempeñar un papel activo debido principalmente a la influencia de nuestro vecino del norte EEUU, que es el mayor impulsor de este proceso. 
\vspace{5mm}

\noindent En este contexto, encontramos que el proceso de formación de los ingenieros en sistemas computacionales debe buscar elevar su calidad en cuanto a desarrollar la capacidad para resolver los problemas que traerán consigo los desafíos de una sociedad en la ardua competencia a la que se verá obligado dado el avance y la consolidación del proceso de globalización.
\vspace{5mm}

\noindent Para ello se debe de transformar el contexto donde se desarrollan, así como los paradigmas que venían apegados a esta profesión, buscando aprender y superar lo que le permea la globalización, para ello se debe buscar que los estudiantes superen la conducta pasiva que se tiene en los países de tercer mundo donde se va únicamente siguiendo la estela que dejan los países desarrollados por medio de una aceptación acrítica de sus medios y procedimientos en busca de que desarrollen actitudes propositivas para estar a la par e incluso a la vanguardia de estos en determinadas áreas. Además, se debe de privilegiar la capacidad de análisis de la realidad, en la cual se aplican los conocimientos también llamados \textit{habilidades duras} de tal forma que tomen conciencia del contexto en el que viven para que busquen mejorar las condiciones que perciban, dado que aquél es distinto del que viven los demás países llegando a ser único incluso dentro de este mundo globalizado \cite{res:2008}.
\vspace{5mm}

\noindent Una vez identificado el contexto en el cual se va a desenvolver el ingeniero en sistemas computacionales y las herramientas con las que va a trabajar, o en general con todo aquello con lo que va a interactuar, nos adentraremos en el llamado \textit{Ecosistema Digital} de nuestro contexto, donde la cantidad de información sobrepasa las capacidades de análisis actuales en pro de la toma de decisiones y el desarrollo tecnológico da pie a otro enfoque en el trabajo de estos personajes. Todo esto dará pie a una serie de retos específicos que irán cambiando constantemente con el desarrollo del área y el avance de la globalización.



\section{Retos del Ingeniero en Sistemas Computacionales}

Dado el panorama en el que nos encontramos, los ingenieros en sistemas computacionales, y en general de cualquier carrera afín a las TICs, enfrentan múltiples retos debido a las exigencias del sector y a un rezago cada más marcado en los distintos programas existentes que dan trompicones buscando ponerse al día pero que los largos procesos burocráticos lo hacen complicado.
\vspace{5mm}

\noindent Se busca que estos profesionales propongan soluciones innovadoras ante las necesidades que aqueja su contexto, no sólo laboral como ya vimos, también social. Sin embargo, tiene un gran apoyo en las distintas tecnologías que se van desarrollando, pero esto le presenta otra nueva problemática al tener que estar al día en todas ellas.
\vspace{5mm}

\noindent Al desenvolverse en su ambiente laboral, el ingeniero en sistemas computacionales debe entender las dinámicas de las empresas para lograr generar valor agregado a las soluciones que tendrá que desarrollar y que éstas se adapten correctamente a lo requerido. Esto implica que además debe ser capaz de analizar la información para la toma de decisiones en un momento dado.
\vspace{5mm}

\noindent También tendrá que trabajar en equipos de trabajo multidisciplinario, realizar planeaciones, dar seguimiento entre otras cosas. Estas habilidades deben ser fortalecidas por conocimiento en las áreas de gerencia de proyectos y de aplicación de marcos económicos; las primeras le facilitarán el trabajar de manera efectiva en equipos de trabajo donde requiere competencias comunicativas, tanto orales como escritas, junto con habilidades de negociación y manejo de grupos que le permitirán navegar a buen recaudo con los equipos que lleve y con los que deba convivir durante el desempeño de sus labores.
\vspace{5mm}

\noindent Es de vital importancia que se tenga capacidad de auto-aprendizaje, pues debe poder apropiarse de los avances tecnológicos y metodológicos que le faciliten cumplir con las necesidades y restricciones de la organización. Lo anterior va aunado al manejo de otros idiomas, principalmente el inglés, considerando que en esta disciplina es el idioma predominante, tanto en textos escritos como en eventos nacionales e internacionales, e incluso para interactuar con personas en cualquier parte del mundo debido a la globalización, de la cual se habló en la sección anterior, ya que se llegará a tener equipos multiculturales con los cuales habrá convivencia que puede ir desde un simple intercambio de información hasta tener que desarrollar un trabajo al completo \cite{redis:2016}.


\subsection{Retos a nivel mundial}

Es evidente cómo el panorama mundial ha impactado en nuestra profesión y en el campo de las TICs en general, pero distintas organizaciones a nivel mundial lo han dejado claro a través de diversas publicaciones o planteamientos que han realizado para encaminar los esfuerzos de nuestra área hacia esos objetivos.

\subsubsection{ONU}
La Organización de las Naciones Unidas (ONU) a través del programa \textit{We can end poverty} propuso los conocidos como \textit{Objetivos del Nuevo Milenio} (ONM), que son ocho propósitos de desarrollo humano fijados en el año 2000, que los 189 países miembros de las Naciones Unidas acordaron conseguir para el año 2015. Entre ellos podemos encontrar el objetivo número ocho que dice: \textquote{Fomentar una Alianza Mundial para el Desarrollo} y en su meta 8.F dice: \textquote{En cooperación con el sector privado, hacer más accesible los beneficios de las nuevas tecnologías, especialmente las de información y comunicaciones}, basados en indicadores como el acceso a internet y a la telefonía móvil principalmente, tanto en dispositivos como en los planes de datos \cite{onu:2000}.
\vspace{5mm}

\noindent A partir de ese año continuó con un nuevo programa llamado \textit{Objetivos de Desarrollo Sostenible}, que en su objetivo número 9 dice: \textquote{Construir infraestructuras resilientes, promover la industrialización inclusiva y sostenible y fomentar la innovación} donde habla de la importancia del progreso tecnológico que debe ser la base para aumentar la eficiencia y mejorar los servicios. Y a lo largo de sus distintas metas la tecnología juega un papel crucial en todas y cada una de ellas \cite{onu:2015}.

\subsubsection{NAE}
La Academia Nacional de Ingeniería (NAE) de los Estados Unidos, determinó que de los catorce \textit{Grandes Desafíos para la Ingeniería} para este siglo XXI, cuatro los plantearía específicamente relacionados con las tecnologías de la información (TIC), y son: 
Avanzar en la informática para la salud 
Proteger el ciberespacio 
Enriquecer la realidad virtual 
Mejorar las herramientas de descubrimiento científico
Sin embargo, entre los demás podemos ver relación con la ingeniería en sistemas de una u otra forma, ya sea por simple mantenimiento o por un desarrollo necesario para esta \cite{nae:2008}.

\subsubsection{UNESCO}
La Organización de las Naciones Unidas para la Educación, la Ciencia y la Cultura (UNESCO) un artículo titulado \textquote{Futuros posibles: Diez tendencias para el siglo XXI}, menciona el Auge de la Tercera Revolución Industrial, en donde el desarrollo acelerado de las TIC proyectan una sociedad programada, promete una sociedad de redes, descentralizada, más democrática, menos jerárquica que traerán consigo la mundialización y tal vez una nueva forma de colonización del planeta \cite{unesco:2002}. 

\subsubsection{OCDE}
La Organización para la Cooperación y el Desarrollo Económico (OCDE), en su documento Perspectivas de las tecnologías de la información 2010, indica que las tecnologías de la información (TI) y el Internet son factores primordiales para la investigación, la innovación, el crecimiento y el cambio social. Esta publicación analiza la crisis económica y la recuperación, e indica que las perspectivas para las industrias de bienes y servicios de TI son positivas tras sortear el periodo de dificultades económicas, lo cual es un mejor escenario que durante la crisis de principios de la década de 2000. La industria sigue reestructurando, ante la presencia de economías no pertenecientes a la OCDE, sobre todo China e India, a los principales proveedores de bienes y servicios relacionados con las tecnologías de la información y las comunicaciones (TIC). Se estudia ampliamente el papel de las TIC en el combate a los problemas ambientales y del cambio climático, y se hace especial hincapié en su rol para permitir la mayor difusión de las mejoras en cuestiones ambientales dentro de las economías y consolidar cambios sistémicos en las conductas. Se abordan las últimas tendencias en las políticas de la OCDE sobre TIC para verificar si durante la recuperación surgen nuevos desafíos. Las prioridades se concentran ahora en lograr que la economía avance, destacar las habilidades y el empleo en materia de TIC, la difusión de la banda ancha, el capital de riesgo y la investigación en el ámbito de las TIC, así como un mayor y nuevo énfasis en el uso de las TIC para superar los problemas ambientales y el cambio climático \cite{oecd:2010,espacios:2018}.



\section{\textit{Habilidades Blandas} en un mundo técnico}

Como se ha visto, los retos son muchos y ambiciosos, se espera mucho del área y eso ha conllevado a que los equipos de trabajo sean más grandes y más polifacéticos, esto ha llevado a que se cambien las reglas de juego en el mercado laboral debido a que se da por sentado que la persona contratada tendrá una capacidad técnica suficiente porque las instituciones han demostrado cumplir en ese campo en una medida suficiente. Sin embargo, la industria ahora hace hincapié en una capacitación orientada a trabajar en equipo, manejo de proyectos y comunicación para que sea posible trabajar en proyectos significativos \cite{innova:2016,IEEEreferencias:Web1}.
\vspace{5mm}

\noindent La enseñanza de estas \textit{habilidades blandas} lleva tiempo, sobre todo porque debe permear en un ámbito donde la mayoría de los estudiantes  no entienden el beneficio de éstas. Este ámbito obsoleto puede verse reflejado en un estudio realizado donde los nuevos profesionales buscaron implementar metodologías ágiles, que dan énfasis en una mayor comunicación e interacción entre todos los miembros del proceso de desarrollo, en las empresas ya establecidas donde laboran profesionales con experiencia y una educación tradicional, acostumbrados a proyectos tipificados como \textit{stand-alone} con menos interacción \cite{IEEEreferencias:Web3}.
\vspace{5mm}

\noindent Estas nuevas habilidades son el principal diferenciador en el nuevo mercado laboral debido a que esto da un mayor rendimiento en su desempeño en la industria, sobre todo en los puestos directivos. El desarrollo de las mismas da beneficios en ambas partes, a la industria un profesional más capacitado y a los profesionales mayores oportunidades de desarrollo y una mejor calidad de vida. 
\vspace{5mm}

\noindent En el momento de optar por el desarrollo de habilidades blandas surgen algunos obstáculos. El primero de ellos se refiere a la percepción sobre la dificultad de definir, enseñar, y evaluar estas competencias. El segundo, a la atención que en la actualidad se está brindando a la tecnología que tiende a desvalorizar su importancia. En algunos casos enseñarlas es considerado como responsabilidad de alguien más, permitiendo que el estudiante reaccione con apatía respecto a la necesidad de contar con estas habilidades. Por otro lado, se evidencia una falta de conexión entre el ámbito de la enseñanza y el ámbito laboral en lo que se refiere a habilidades blandas. La comunidad laboral busca empleados que cuenten con estas habilidades, mientras que los educadores no les atribuyen suficiente importancia \cite{pcbee:2000}.
\vspace{5mm}

\noindent Pero ya se ha visto que estas habilidades resultan medulares para afrontar los nuevos retos que surgen y los que seguramente seguirán surgiendo conforme el área adquiere mayor importancia y desarrollo. Lo que nos lleva a otra cuestión, para hacer frente, ¿qué habilidades son las que deberían desarrollarse con mayor énfasis?
\vspace{5mm}

\noindent Esta pregunta nuevamente tiene muchas respuestas, es difícil desarrollar todas las habilidades pero también es cierto que no es necesario hacerlo porque dependerá en gran medida de en qué trabajamos, el puesto que ocupemos y en general el contexto en el que nos desenvolvemos. Sin embargo, se han hecho estudios en la industria sobre las necesidades de la misma, uno de ellos es el presentado en \textit{International Journal of Business and Social Science} que menciona que la habilidad más demandada para los programadores es la \textbf{resolución de problemas} ya que es necesaria para el desarrollo y el mantenimiento del software, un nivel por debajo se encuentra el trabajo en equipo, la capacidad de escuchar, la adaptabilidad a nuevas tecnologías y lenguajes, visualización y conceptualización, administración del tiempo, la capacidad de transformar lo aprendido en algo práctico, ser multitareas y la comunicación verbal. 
\vspace{5mm}

\noindent Todo esto nos permite entrever que el profesional va a trabajar en un ámbito de equipo, seguramente muy grandes por el tamaño y la complejidad de los sistemas, y para ello debe ser un miembro del equipo efectivo, capaz de usar las habilidades citadas en su contexto. Además, un programador debe poder fusionar y aplicar las especificaciones del usuario con sus conocimientos de negocios, hardware y software existente. Los programadores son responsables de las aplicaciones informáticas compuestas por una multitud de módulos, muchos de los cuales deben completarse en una secuencia particular. Cuando los módulos se retrasan, el tiempo de finalización de un proyecto completo se ve afectado. Con frecuencia, los componentes de la aplicación se escribirán en varios idiomas. Además, las tecnologías de hardware y software cambian rápidamente. Por lo tanto, para tener éxito, los programadores deben adaptarse rápidamente. \cite{ijbss:2014}
\vspace{5mm}

\noindent Al final, el desarrollo de estas \textit{habilidades blandas} da pie a que podamos sobrepasar los retos ya sea por su complejidad o por tener una visión distinta que antaño no sería posible por la carencia de las mismas. El poder trabajar en equipos grandes que nos permiten sobrellevar la gran carga, tener un punto de vista de alguien que no sabe nada de nuestra área, evitar trabajar de más por malentendidos que ahora podremos resolver son algunas de las cosas que nos brinda este nuevo paradigma en el desempeño de nuestras funciones.